\documentclass[11pt,a4paper]{moderncv}
%\moderncvtheme[blue]{casual}
\moderncvtheme[blue]{classic}
\usepackage[utf8]{inputenc}

% adjust the page margins
\usepackage[scale=0.8]{geometry}
%\setlength{\hintscolumnwidth}{3cm}						% if you want to change the width of the column with the dates
%\AtBeginDocument{\setlength{\maketitlenamewidth}{6cm}}  % only for the classic theme, if you want to change the width of your name placeholder (to leave more space for your address details
\AtBeginDocument{\recomputelengths}                     % required when changes are made to page layout lengths

% personal data
\firstname{\huge Cristián D. Maureira Fredes}
\familyname{}
\title{\large Computer Science Engineer and M.Sc. Student}               % optional, remove the line if not wanted
\address{San Guillermo \#852, Placeres}{Valparaíso, Chile}    % optional, remove the line if not wanted
\mobile{+56 9 957 99 209}                    % optional, remove the line if not wanted
%\phone{+56 32 232232}                      % optional, remove the line if not wanteD
%\fax{fax (optional)}                          % optional, remove the line if not wanted
\email{cmaureir@csrg.inf.utfsm.cl}                      % optional, remove the line if not wanted
%\extrainfo{asdf}    				% optional, remove the line if not wanted
\photo[64pt]{photo/me}               % '64pt' is the height the picture must be resized to and 'picture' is the name of the picture file; optional, remove the line if not wanted
%\quote{Curriculum Vitae}                 % optional, remove the line if not wanted
%\nopagenumbers{}                             % uncomment to suppress automatic page numbering for CVs longer than one page
%content
\begin{document}
\maketitle

\section{Education}
\cventry{2011 to present}
	{M.Sc. Computer Science}
	{Universidad Técnica Federico Santa María}
	{Valparaíso}
	{\emph{in progress}}
	{1st Semester}
\cventry{2006 to present}
	{Computer Science Engineering}
	{Universidad Técnica Federico Santa María}
	{Valparaíso}
	{\emph{in progress}}
	{10th Semester}
\cventry{1993-2005}
	{Elementary and High School}
	{Instituto del Puerto}
	{San Antonio, Valparaíso}
	{}{}

\vspace{-0.5cm}
\section{Experience}
\subsection{Vocational}

\cventry{2011 to present}
    {CTI-HPC Research Assistant}
    {Researching GPU topics applied to Astronomical simulations}
    {Center for Technological Innovation in High Performance Computing (CTI-HPC)}
    {UTFSM}
    {}

\cventry{2011 to present}
	{ALMA-UTFSM Representative}
	{In charge of the national and international representation, coordinating the activities, work and projects inside the group.}
	{Computer Systems Research Group (CSRG)}
	{UTFSM}
	{}

\cventry{2010 March to present}
	{ARM Research Group Member}
	{Research ARM architecture systems for the development of micro computers with low energy consumption}
	{Computer Systems Research Group (CSRG)}
	{UTFSM}
	{}

\cventry{2010 March to present}
	{Presence Assistantship}
	{Scientific Computation I and II}
	{Computer Science Department, UTFSM}
	{}{}

\cventry{2009 July to present}
	{ALMA-UTFSM Research Assistantship}
	{Project Leader of SoftWare development, Analysis and Testing (SWAT) Team, in charge of the development of the software at the ALMA-UTFSM group.
	Maintenance and development of the Sampling System project, build upon the ALMA Common Software framework.
	Also, a member of Alarms Configuration GUI, with the propose of develop a Graphic User Interface to configure the ACS Alarms System}
	{Computer Systems Research Group (CSRG)}
	{UTFSM}
	{}

\cventry{2008 to 2009}
	{Project Leader}
	{$\mu$Bot Interface, A Generic Graphic User Interface to robotic platform control}
	{A ``Vortex Labs'' project, http://www.vortexlabs.cl/}
	{PIE>A, Sponsored Project, UTFSM, http://www.piea.usm.cl}
	{Computer Systems Research Group (CSRG)}

\cventry{2008 August to 2009 July}
	{ALMA-UTFSM Research Assistantship}
	{Member of SoftWare development, Analysis and Testing (SWAT) Team, in charge of the development of the software at the ALMA-UTFSM group.
	Maintenance and development of the Sampling System project, build upon the ALMA Common Software framework.
	Also, a member of Repackaging ACS for Embedded Systems, with the propose of separate the run-time and development
	proper of this build environment}
	{Computer Systems Research Group (CSRG)}
	{UTFSM}
	{}

\cventry{2008 to 2009}
	{Project Member}
	{Rescue Live CD, A Fedora official Spin release}
	{PIE>A, Sponsored Project, UTFSM}
	{Computer Systems Research Group (CSRG)}
	{http://www.piea.usm.cl}

\cventry{2008 March to December}
	{Presence Assistantship}
	{Fundamentos de la Informática  I and II}
	{Computer Science Department, UTFSM}
	{}{}

\cventry{2007 to 2008}
	{Computer Laboratory Assistantship}
	{UTFSM,Valparaíso}
	{Member of the Computer Laboratory (LabComp), laboratory in charge of all the Computer Science Student's services, with the task of system administrator, networking, scripting development, and teaching Basic Linux courses}
	{http://labcomp.inf.utfsm.cl}{}

\cventry{2007 July - December}
	{Programming Laboratory Assistantship}
	{C language}
	{Computer Science Department, UTFSM}
	{Valparaíso}
	{}

\subsection{Internships}

\cventry{2010}{Professional Internship}
	{Universidad Técnica Federico Santa María (UTFSM)}
	{Computer Systems Research Group (CSRG)}
	{Working on the implementation of an alternative to the ALMA Common Software (ACS) Notification Channel using OpenDDS, http://www.opendds.org/}
	{Valparaíso, Chile}
\cventry{2009}{Professional Internship}
	{Associated Universities Incorporated (AUI), Joint ALMA Observatory}
	{Working on the Java implementation of the ACS Basic Control Interface Specification; Developing a GUI tool to perform test in the Antenna Bus Master (ABM) and Teaching a Python Course to engineers and scientific people working on the ALMA project.}
	{Operations Support Facility (OSF)}
	{Atacama, Chile}


\subsection{Miscellaneous}
	\cventry{2010}
		{Technical Organization Team Member}
		{PASI 2011}
		{Pan-American Advance Studies Institute}
		{http://www.bu.edu/pasi/}
		{}

	\cventry{2009}
		{Co-director and founder}
		{Arch Linux Chile}
		{Arch Linux chilean user community}
		{http://www.archlinux.cl}
		{}
	\cventry{2007 to 2009}
		{Organization Team Member}
		{Computer Science Promote}
		{Computer Science Department, UTFSM}
		{http://promocion.inf.utfsm.cl}
		{}
	\cventry{2008 to 2009}
		{Organization Team Member}
		{Computer Science Talks Series}
		{http://cci.inf.utfsm.cl}
		{}{}
	\cventry{2007 to 2008}
		{Coordination}
		{Technical Talks Series}
		{Computer Science Department, UTFSM}
		{}{}
	\cventry{2007 October}{Organization Team Member}{Feria de Software}{http://www.feriasoftware.cl/}{}{}


\section{Languages}
\cvlanguage{Spanish}{Native}{Speaks, Reads, Writes, Technical and Colloquial}
\cvlanguage{English}{Medium}{Speaks, Reads, Writes}


\section{Computer Systems skills}
\cvcomputer     {Operating System (Linux)}{Arch Linux, Fedora, CentOS, Scientific Linux, Ubuntu}
				{Operating system (Microsoft)}{Windows XP, Vista, 7}
\cvcomputer     {Languages}{C, C++, Python, Java, Bash, PHP}
				{DBA}{postgreSQL, MySQL}
\cvcomputer		{SysAdmin}{Web (Httpd, PHP), Software Configuration Management (Git, SVN, CVS), Email (Sendmail),LDAP (OpenLDAP)}
				{Web Development}{PHP, CSS, HTML}
\newpage
\section{Events attendance}
\cventry{2011}{HPC School}{High Performance Computing School}{CMCC, Universidad de la Frontera (UFRO)}{Temuco, Chile}{http://www.cmcc.ufro.cl/hpc2011/}
\cventry{2010}{PASI}{Pan-American Advanced Studies Institute}{Scientific Computing in the Americas: the challenge of massive parallelism}{UTFSM, Valparaíso, Chile}{http://www.bu.edu/pasi/}
\cventry{2009}{EVIC}{VI Escuela de Verano Latino-americana en Inteligencia Computacional, III Encuentro de Verano Latino-americana en Robótica}{Universidad de Chile}{Santiago}{}
\cventry{2009}{XVIII Feria de Software}{Presentation of $\mu$Bot Interface project}{UTFSM}{Valparaíso}{http://vortexlabs.cl}
\cventry{2009}{ACS Workshop}{Attendance to the general meeting and final course (Advanced)}{UTFSM}{Valparaíso}{}
\cventry{2009}{SCAT}{Scientific Computing Advanced Training}{Final SCAT Meeting \& Satellite School}{UTFSM, Chile}{}
\cventry{2009}{Encuentro Nacional de Computación}{Jornadas Chilenas de la Computación}{}{}{http://jcc2009.usach.cl/}
\cventry{2008 and 2009}{FLiSol}{Latin American Free Software Installation Festival}{Valparaíso}{}{}
\cventry{2008}{ACS Workshop}{Attendance to the general meeting and final course (Basic)}{UTFSM}{Valparaíso}{}
\cventry{2007, 2008 and 2009}{Encuentro Linux}{National gathering of Linux users}{Chile}{}{}

\section{Interests and Hobbies}
\cvlistdoubleitem[\Neutral]{Playing Online Games}{Design and Development of Websites}
\cvlistdoubleitem[\Neutral]{Graphic Design in general}{Play some instruments (Guitar, Flute)}
\cvlistdoubleitem[\Neutral]{Role Playing}{Movies in general, dvd, cinema, etc}
\cvlistdoubleitem[\Neutral]{Coin Collection}{Music Composition}


\renewcommand{\listitemsymbol}{-} % change the symbol for lists

%\section{Further Personal Information}
%\cvcomputer     {Birthday}{19 May, 1988}        {Age}{20}
%\cvcomputer     {RUN}{16.759.352-6}             {Civilian State}{Single}
%\cvcomputer     {Nationality}{Chilean}          {Homepage}{\url{http://saint.lapalta.net}}

\section{Honor and Awards}
\cvlistitem{M.Sc. Financial help, due to good academic performance.}
\cvlistitem{M.Sc. Fee payment exemption scholarship 100\%, due to good academic performance.}
\cvlistitem{Be a part of the, UTFSM \emph{Honor List}, to the best academic performance, 2007, 2008, 2009 and 2010}
\cvlistitem{First Place in Academic Performance, Instituto del Puerto (high school), '05 generation}
\newpage
\section{References}
\cventry{}{Horst Von Brand}{Central Department of Computational Services Director}{Universidad Técnica Federico Santa María}{\url{vonbrand@inf.utfsm.cl}}{}
\cventry{}{Cecilia Reyes}{Computer Science Department Adjoint Professor}{Universidad Técnica Federico Santa María}{\url{reyes@inf.utfsm.cl}}{}

%\section{Publications}
% Publications from a BibTeX file
\nocite{*}
\bibliographystyle{plain}
\bibliography{latex/pub-en}       % 'publications' is the name of a BibTeX file

\end{document}
