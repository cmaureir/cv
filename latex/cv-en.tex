\documentclass[11pt,a4paper]{moderncv}
%\moderncvtheme[blue]{casual}
\moderncvtheme[blue]{classic}
\usepackage[utf8]{inputenc}
\newcommand{\myname}{\textbf{Cristián Maureira}}
\usepackage[scale=0.8]{geometry}
%\setlength{\hintscolumnwidth}{3cm}
%\AtBeginDocument{\setlength{\maketitlenamewidth}{6cm}}
\AtBeginDocument{\recomputelengths}

%
% Personal Information
%
\firstname{Cristián}
\familyname{Maureira}
\title{\large Computer Science Engineer and M.Sc. Student}
\address{Carmen \#426, Cerro Placeres}{Valparaiso, Chile}
\mobile{+56 9 957 99 209}
%\phone{+56 32 232232}
%\fax{fax (optional)}
\email{cmaureir@hpc.cl}
%\extrainfo{asdf}
\photo[64pt]{photo/me}
%\quote{Curriculum Vitae}
%\nopagenumbers{}

%
% Document
%
\begin{document}
\maketitle

%
% Education
%
\section{Education}

\cventry{2011 to present}
        {M.Sc. Computer Science}
        {Universidad Técnica Federico Santa María}
        {Valparaiso}
        {\emph{(in progress)}}
        {3rd Semester}
\cventry{2006 to present}
        {Computer Science Engineering}
        {Universidad Técnica Federico Santa María}
        {Valparaiso}
        {\emph{(in progress)}}
        {12th Semester}
\cventry{1993-2005}
        {Elementary and High School}
        {Instituto del Puerto}
        {San Antonio, Valparaiso}
        {}{}

\vspace{-0.5cm}

%
% Experience
%
\section{Experience}
\subsection{Vocational}

\cventry{May 2012 to November 2012}
        {Master Student}
        {Master thesis development, which consist in a %
        ``Semi-Keplerian direct-summation $N$-body integrator based on %
        GPU Computing'' }
        {Max-Planck-Institut für Gravitationphysik (Albert-Einstein-Institute)}
        {}
        {\url{http://aei.mpg.de}}


\cventry{2011 to 2012}
        {CTI-HPC Research Assistant}
        {Researching Astronomical simulations using GPU technologies, %
        and parallel computing projects in general.}
        {Center for Technological Innovation in High Performance Computing (CTI-HPC)}
        {UTFSM}
        {\url{http://hpc.cl}}

\cventry{2011 to 2012}
        {ALMA-UTFSM Representative}
        {In charge of the national and international representation, coordinating the activities, work and projects inside the group.}
        {Computer Systems Research Group (CSRG)}
        {UTFSM}
        {\url{http://csrg.inf.utfsm.cl/proyectos/alma-utfsm/}}

\cventry{2010 to 2011}
        {ARM Research Group Member}
        {Research ARM architecture systems for the development of micro computers with low energy consumption}
        {Computer Systems Research Group (CSRG)}
        {UTFSM}
        {\url{http://csrg.inf.utfsm.cl/proyectos/arm/}}

\cventry{2010 to 2011}
        {Presential Assistantship}
        {Courses: Scientific Computation I and II}
        {Computer Science Department, UTFSM}
        {}{}

\cventry{2009 July to 2012}
        {ALMA-UTFSM Research Assistantship}
        {Project Leader of Software development, Analysis and Testing (SWAT) Team, in charge of the development of the software at the ALMA-UTFSM group.
        Maintenance and development of the Sampling System project, build upon the ALMA Common Software framework.
        Also, a member of Alarms Configuration GUI, with the propose of develop a Graphic User Interface to configure the ACS Alarms System}
        {Computer Systems Research Group (CSRG)}
        {UTFSM}
        {\url{http://csrg.inf.utfsm.cl/proyectos/alma-utfsm/}}

\cventry{2008 to 2009}
        {Project Leader}
        {$\mu$Bot Interface, A Generic Graphic User Interface to robotic platform control}
        {A ``Vortex Labs'' project}
        {PIE>A, Sponsored Project, UTFSM, http://www.piea.usm.cl}
        {Computer Systems Research Group (CSRG)}

\cventry{2008 August to 2009 July}
        {ALMA-UTFSM Research Assistantship}
        {Member of Software development, Analysis and Testing (SWAT) Team, in charge of the development of the software at the ALMA-UTFSM group.
        Maintenance and development of the Sampling System project, build upon the ALMA Common Software framework.
        Also, a member of Repackaging ACS for Embedded Systems, with the propose of separate the run-time and development
        proper of this build environment}
        {Computer Systems Research Group (CSRG)}
        {UTFSM}
        {\url{http://csrg.inf.utfsm.cl/proyectos/alma-utfsm/}}

\cventry{2008 to 2009}
        {Project Member}
        {Rescue Live CD, A Fedora official Spin release}
        {PIE>A, Sponsored Project, UTFSM}
        {Computer Systems Research Group (CSRG)}
        {\url{http://www.piea.usm.cl}}

\cventry{2008 March to December}
        {Presential Assistantship}
        {Courses: Informatics Fundamentals  I and II}
        {Computer Science Department, UTFSM}
        {}{}

\cventry{2007 to 2008}
        {Computer Laboratory Assistantship}
        {UTFSM,Valparaiso}
        {Member of the Computer Laboratory (LabComp), laboratory in charge of all the Computer Science Student's services, performing the tasks of system administrator, networking, scripting development, and teaching Linux courses}
        {\url{http://labcomp.inf.utfsm.cl}}{}

\cventry{2007 July to December}
        {Programming Laboratory Assistantship}
        {C language}
        {Computer Science Department, UTFSM}
        {Valparaiso}
        {}

\subsection{Internships}

\cventry{2011}{Professional Internship}
        {Universidad Técnica Federico Santa María (UTFSM)}
        {Center for Technological Innovation in High Performance Computing (CTI-HPC)}
        {Working on the implementation of a gravitational n-body simulation, using different parallel computing approaches, as POSIX threads, OpenMP and GPU Computing.}
        {Valparaiso, Chile, \url{http://hpc.cl}}

\cventry{2010}{Professional Internship}
        {Universidad Técnica Federico Santa María (UTFSM)}
        {Computer Systems Research Group (CSRG)}
        {Working on the implementation of an alternative to the ALMA Common Software (ACS) Notification Channel using OpenDDS, \url{http://www.opendds.org/}}
        {Valparaiso, Chile, \url{http://csrg.cl}}

\cventry{2009}{Professional Internship}
        {Associated Universities Incorporated (AUI), Joint ALMA Observatory}
        {Working on the Java implementation of the ACS Basic Control Interface Specification; Developing a GUI tool to perform test in the Antenna Bus Master (ABM) and Teaching a Python Course to engineers and scientific people working on the ALMA project.}
        {Operations Support Facility (OSF)}
        {Atacama, Chile, \url{http://www.almaobservatory.cl}}


\subsection{Miscellaneous}

\cventry{2010}
        {Technical Organization Team Member}
        {PASI 2011}
        {Pan-American Advance Studies Institute}
        {\url{http://www.bu.edu/pasi/}}
        {}

\cventry{2009}
        {Co-director and founder}
        {Arch Linux Chile}
        {Arch Linux Chilean user community}
        {\url{http://www.archlinux.cl}}
        {}

\cventry{2007 to 2009}
        {Organization Team Member}
        {Computer Science Promote}
        {Computer Science Department, UTFSM}
        {\url{http://promocion.inf.utfsm.cl}}
        {}

\cventry{2008 to 2009}
        {Organization Team Member}
        {Computer Science Talks Series}
        {\url{http://cci.inf.utfsm.cl}}
        {}{}

\cventry{2007 to 2008}
        {Coordination}
        {Technical Talks Series}
        {Computer Science Department, UTFSM}
        {}{}

%\cventry{2007 October}
%        {Organization Team Member}
%        {UTFSM Software Fair}
%        {\url{http://www.feriasoftware.cl/}}
%        {}{}

\section{Languages}
\cvlanguage{Spanish}{Native}{Speaks, Reads, Writes, Technical and Colloquial}
\cvlanguage{English}{Medium}{Speaks, Reads, Writes}

\section{Computer Systems skills}
\cvcomputer{Programming Languages}{C/C++, Python, Java}
           {Parallel Computing}{CUDA, MPI, OpenMP, POSIX threads}
\cvcomputer{Operating Systems}{Linux Distributions}
           {Miscellaneous}{ZeroMQ, CUBLAS, Numpy}
\cvcomputer{System Administration}{Apache, PHP, Git, SVN)}
           {Web Development}{PHP, CSS, MySQL, HTML}

\section{Events attendance}
%\cventry{2012}
%        {SFB Meeting}
%        {}
%        {}
%        {}
%        {}

\cventry{2012}
        {Astro-GR}
        {GraviDy: A modular direct $N$-body GPU integrator}
        {National Astronomical Observatories (NAOC) of Chinese Academy of Sciences
        (CAS)}
        {Beijing, China}
        {\url{http://astro-gr.aei.mpg.de/Astro-GR@Beijing-2012}}

\cventry{2011}
        {First Argentine GPGPU Computing School for Scientific Applications}
        {Facultad de Matemáticas, Astronomía y Física}
        {Universidad Nacional de Córdoba, Argentina}
        {\url{http://www.famaf.unc.edu.ar/grupos/GPGPU/EscuelaGPGPU2011/}}
        {}

\cventry{2011}
        {HPC School}
        {High Performance Computing School}
        {CMCC, Universidad de la Frontera (UFRO)}
        {Temuco, Chile}
        {\url{http://www.cmcc.ufro.cl/hpc2011/}}

\cventry{2010}
        {PASI}
        {Pan-American Advanced Studies Institute}
        {Scientific Computing in the Americas: the challenge of massive parallelism}
        {UTFSM, Valparaiso, Chile}
        {\url{http://www.bu.edu/pasi/}}

\cventry{2009}
        {EVIC}
        {VI Escuela de Verano Latino-americana en Inteligencia Computacional, %
        III Encuentro de Verano Latino-americana en Robótica}
        {Universidad de Chile}
        {Santiago}
        {}

\cventry{2009}
        {XVIII Feria de Software}
        {Presentation of $\mu$Bot Interface project}
        {UTFSM}
        {Valparaiso}
        {\url{http://vortexlabs.cl}}

\cventry{2009}
        {ACS Workshop}
        {Attendance to the general meeting and final course (Advanced)}
        {UTFSM}
        {Valparaiso}
        {}

\cventry{2009}
        {SCAT}
        {Scientific Computing Advanced Training}
        {Final SCAT Meeting \& Satellite School}
        {UTFSM, Chile}
        {}

\cventry{2009}
        {Encuentro Nacional de Computación}
        {Jornadas Chilenas de la Computación}
        {}
        {}
        {\url{http://jcc2009.usach.cl/}}

\cventry{2008 and 2009}
        {FLiSol}
        {Latin American Free Software Installation Festival}
        {Valparaiso}
        {}
        {}

\cventry{2008}
        {ACS Workshop}
        {Attendance to the general meeting and final course (Basic)}
        {UTFSM}
        {Valparaiso}
        {}

\cventry{2007, 2008 and 2009}
        {Encuentro Linux}
        {National gathering of Linux users}
        {Chile}
        {}
        {}

%\section{Interests and Hobbies}
%\cvlistdoubleitem[\Neutral]{Playing Online Games}
%                           {Design and Development of Websites}
%\cvlistdoubleitem[\Neutral]{Graphic Design in general}
%                           {Play some instruments (Guitar, Flute)}
%\cvlistdoubleitem[\Neutral]{Role Playing}
%                           {Movies in general, dvd, cinema, etc}
%\cvlistdoubleitem[\Neutral]{Coin Collection}
%                           {Music Composition}

\renewcommand{\listitemsymbol}{-}


\section{Honor and Awards}

\cvlistitem{Scholarships for Master's Studies in Chile (2012) given by the %
            ``Comisión Nacional de Investigación Científica y Tecnológica  %
            (CONICYT)''}

\cvlistitem{Initiation in Scientific Research Scholarship (PIIC) given by %
            ``Dirección General de Investigación y Postgrado'', UTFSM}

\cvlistitem{Master's Studies Financial help, due to good academic performance.}

\cvlistitem{Master's Studies Fee payment exemption scholarship 100\%, due to good %
            academic performance.}

\cvlistitem{Be a part of the, UTFSM \emph{Honor List}, to the best academic %
            performance, 2007, 2008, 2009, 2010 and 2011.}

\cvlistitem{First Place in Academic Performance, Instituto del Puerto %
            (high school), '05 generation.}

\section{References}
    \cventry{}
            {Pau Amaro-Seoane}
            {Senior Scientist}
            {Max-Planck-Institut für Gravitationphysik (Albert-Einstein-Institut)}
            {\url{Pau.Amaro-Seoane@aei.mpg.de}}
            {}

    \cventry{}
            {Horst Von Brand}
            {Central Department of Computational Services Director}
            {Universidad Técnica Federico Santa María}
            {\url{vonbrand@inf.utfsm.cl}}
            {}

    \cventry{}
            {Cecilia Reyes}
            {Computer Science Department Adjoint Professor}
            {Universidad Técnica Federico Santa María}
            {\url{reyes@inf.utfsm.cl}}
            {}

%\section{Publications}
% Publications from a BibTeX file
\nocite{*}
\bibliographystyle{unsrt}
\bibliography{latex/pub-en}
\end{document}
